%%% A template for a simple PDF/A file like a stand-alone abstract of the thesis.

\documentclass[12pt]{report}

\usepackage[a4paper, hmargin=1in, vmargin=1in]{geometry}
\usepackage[a-2u]{pdfx}
\usepackage[utf8]{inputenc}
\usepackage[T1]{fontenc}
\usepackage{lmodern}
\usepackage{textcomp}

\begin{document}

%% Do not forget to edit abstract.xmpdata.

%% EN
We develop a method to train a document embedding model with an unlabeled
dataset and low computational resources. Using teacher-student training, we
distill SBERT's capacity to capture text structure and Paragraph Vector's
ability to encode extended context into the resulting embedding model. We test
our method on Longformer, a Transformer model with sparse attention that can
process up to 4096 tokens. We explore several loss functions for the
distillation of knowledge from the two teachers (SBERT and Paragraph Vector) to
our student model (Longformer). Throughout experimentation, we show that
despite SBERT's short maximum context, its distillation is more critical to the
student's performance. However, as we also demonstrate, the student model can
benefit from both teachers. Our method improves Longformer's performance on
eight downstream tasks, including citation prediction, plagiarism detection,
and similarity search. Our method shows exceptional performance with few
finetuning data available, where the trained student model outperforms both
teacher models. By showing consistent performance of differently configured
student models, we demonstrate our method's robustness to various changes and
suggest areas for future work.

%% CS
V této práci představujeme metodu strojového učení modelů emedující dokumenty,
která není náročná na výpočetní zdroje ani nevyžaduje anotovaná trénovací data.
S přístupem učitele a studenta, distilujeme kapacitu SBERTa zaznamenat
strukturu textu a schopnost Paragraph Vektoru zpracovat dlouhé dokumenty do
našeho výsledného embedovacího modelu. Naší metodu testujeme na Longformeru,
Transformeru s řídkou attention vrstvou, který je schopný zpracovat dokumenty
dlouhé až 4096 tokenů. Prozkoumáme několik ztrátových funkcí, které nutí
studenta (Longformera) napodobovat výstupy obou učitelů (SBERTa a Paragraph
Vektoru). V experimentech ukazujeme, že i přes omezený kontext SBERTa, je
distilace jeho výstupů pro výkon studenta zásadnější. Nicméně, také ukazujeme,
že student získává prospěch z obou učitelů. Naše metoda dokáže vylepšit
výsledek Longformera na osmi úlohách, které zahrnují predikci citace, detekci
plagiarismu i vyhledávání na základě podobnosti dokumentů. Naše metoda se navíc
ukazuje jako obzvláště účinná v situacích s málo dotrénovávacími daty, kde námi
natrénovaný student překoná i oba učitele. Podobným výkonem odlišně
natrénovaných studentů ukazujeme, že naše metoda je robustní vůči různým
změnám, kde dále navrhujeme možné oblasti budoucího výzkumu.

\end{document}
